\chapter{解析手法}
\label{chap:simulation}

\section{はじめに}
陰解法を用いた場合,行列の反転,すなわち連立一次方程式を解く必要がある.
連立一次方程式の解法には直接法と反復法がある.直接法では厳密に解が求めら
れるが,計算量が多い.反復法は反復計算により近似解を求める方法であり,係
数行列の特性に応じて,高速に解を得られることがある.
CFDでは反復法が多く用いられる.行列の特性によって収束の速度に違いが出る
と考えられるので,各手法を適用し,最適なものを選択する.ここでは次の式で
表される$n$次元の連立一次方程式を解くことを考える.
\begin{equation}
 A\bm{x}=\bm{b}
\label{eq:linear}
\end{equation}
$A$は$n\times n$の係数行列,$x$は$n\times1$の解ベクトル,$b$は
$n\times1$のベクトルを表す.

\section{XX法}